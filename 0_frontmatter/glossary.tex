%This file defines all of the glossary terms.

\makeglossaries

%GLOSSARY TERMS
\newglossaryentry{angstrom}{name={\AA\ },description={a unit of length measuring $10^{-10}$ meters}}
\newacronym{AR}{AR}{active region}
\newglossaryentry{activeregion}{name={active region},description={magnetic structures often associated with flaring sunspot groups that usually have loop structures reaching into the corona}}
\newacronym{CCD}{CCD}{charge-coupled device}
\newacronym{CC}{CC}{correlation coefficient}
\newacronym{CME}{CME}{coronal mass ejection}
\newacronym{FITS}{FITS}{Flexible Image Transport System}
\newglossaryentry{fluxrope}{name={flux rope},description={a twisted collection of flux tubes}}
\newacronym{HMI}{HMI}{Helioseismic Magnetic Imager}
\newacronym{HWHM}{HWHM}{half width at half maximum}
\newacronym{HXR}{HXR}{hard X-ray}
\newacronym{LCT}{LCT}{local correlation tracking}
\newacronym{LCP}{LCP}{left-hand-circularly-polarised}
\newacronym{LOS}{LOS}{line-of-sight}
\newacronym{LTE}{LTE}{local thermodynamic equilibrium}
\newacronym{MDI}{MDI}{Michelson Doppler Imager}
\newacronym{MHD}{MHD}{magnetohydrodynamics}
\newacronym{PFSS}{PFSS}{potential field source surface}
\newacronym{protonproton}{PP}{proton-proton}
\newacronym{PSL}{PSL}{polarity separation line}
\newglossaryentry{quietsun}{name={quiet-Sun},description={regions on the solar surface characterised by weak magnetic fields and a lack of eruptive activity},plural={quiet Sun}}
\newacronym{RCP}{RCP}{right-hand-circularly-polarised}
\newacronym{SDO}{SDO}{Solar Dynamics Observatory}
\newacronym{SEP}{SEP}{solar energetic particle}
\newacronym{SFT}{SFT}{surface flux transport}
\newacronym{SHILLELAgh}{SHILLELAgh}{Solar Wind-Heliosperic Imaging in Latitude and Longitude by Estimating Large-scale Attributes}
\newacronym{SMART}{SMART}{SolarMonitor Active Region Tracker}
\newacronym{SOHO}{SOHO}{Solar and Heliospheric Observatory}
\newacronym{STEREO}{STEREO}{Solar-Terrestrial Relations Observatory}
\newacronym{SXR}{SXR}{soft X-ray}
\newglossaryentry{wlsg}{name={$WL_{SG}$},description={a sum of the horizontal gradient along a detected PSL}}
\newglossaryentry{rvalue}{name={R-value},description={the total magnetic flux near a detected PSL}}

\newacronym{GOES}{GOES}{Geostationary Operational Environmental Satellites}
\newacronym{GSFC}{GSFC}{Goddard Space Flight Center}
\newacronym{SPoCA}{SPoCA}{Spatial Possibilistic Clustering Algorithm}
\newacronym{EUV}{EUV}{extreme ultraviolet}
\newacronym{STARA}{STARA}{Sunspot Tracking And Recognition Algorithm}
\newacronym{ASAP}{ASAP}{Automated Solar Activity Prediction}
\newacronym{RHESSI}{RHESSI}{Reuven Ramaty High Energy Solar Spectroscopic Imager}
\newglossaryentry{cgs}{name=cgs,description={the unit system using centimeters, grams, and seconds for length, mass, and time}}
\newacronym{FIP}{FIP}{first ionisation potential}
\newacronym{EIT}{EIT}{Extreme ultraviolet Imaging Telescope}
\newacronym{BCL}{BCL}{bipole connecting line}
\newacronym{FWHM}{FWHM}{full-width at half-max}
\newacronym{IDL}{IDL}{Interactive Data Language}
\newglossaryentry{umbra}{name=umbra,description={the dark center of a sunspot where the magnetic fields are nearly vertical at the solar surface},plural=umbrae}
\newglossaryentry{penumbra}{name=penumbra,description={the boundary of a sunspot where the magnetic fields are highly inclined to the solar surface and exhibit a filamentary structure},plural=penumbrae}
\newacronym{NGDC}{NGDC}{National Geophysical Data Center}
\newacronym{NOAA}{NOAA}{National Oceanic and Atmospheric Association}
\newglossaryentry{fluxtube}{name={flux-tube},description={an elongated structure with a magnetic field along the longer axis, strongest at the center and dying off rapidly at the edges},plural={flux-tubes}}
\newglossaryentry{SSWIDL}{name={SSWIDL},description={a repository of routines written in IDL generally pertaining to the analysis of solar data},plural={flux-tubes}}

%\newacronym[\glsshortpluralkey=cas,\glslongpluralkey=contrived
%acronyms]{aca}{aca}{a contrived acronym}

% required to print nomenclature name to page header
%\markboth{\MakeUppercase{\nomname}}{\MakeUppercase{\nomname}}


% ----------------------- contents from here ------------------------