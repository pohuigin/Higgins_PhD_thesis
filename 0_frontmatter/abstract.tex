
% Thesis Abstract -----------------------------------------------------


%\begin{abstractslong}    %uncommenting this line, gives a different abstract heading
\begin{abstracts}        %this creates the heading for the abstract page
Sunspots are regions of decreased brightness in the photosphere, or surface layer of the Sun, associated with strong magnetic fields. They are thought to be manifestations of an interior magnetic dynamo and are associated with activity in the atmosphere such as flaring and coronal mass ejections. Sunspots often emerge in groups with life-times on the order of days to weeks and their global emergence rate is characterized by an 11-year cycle. Additionally, they play an integral role in the 22-year magnetic cycle, and can be used as a tracer to study the internal magnetic dynamo. %There are many outstanding questions regarding the birth evolution and death of sunspot groups. %Flare productive sunspot groups tend to be large and complicated, but the precise mechanisms that lead to flaring are not known. The sub-surface structure and mechanisms leading to sunspot emergence are not well understood. The global field has been modeled in a variety of ways, but a direct connection between the observed fields of sunspot groups and the global magnetic field has not been made.\\

To study sunspot groups over an entire solar cycle, we have developed the SolarMonitor Active Region Tracker (SMART) that applies a series of simple image processing techniques to photospheric magnetograms to detect, characterise, and track features manifested in the photospheric field.

This thesis addresses three questions:
\begin{itemize}
\item \emph{What are the conditions in sunspot groups that result in solar flares?} A statistical study of the relationship between the properties of detected sunspot groups and the magnitude of their associated flares has allowed us to determine minimum values of properties allowing for the production of different flare magnitudes. Physical properties related to strong gradients between oppositely oriented magnetic fields are shown to be good indicators for a sunspot group's potential for flaring. 
\item \emph{What mechanisms determine the configuration of the global magnetic field and how does this relate to the solar dynamo?} The properties of detected sunspot groups are compared between solar cycle phases to the global magnetic field configuration. This study shows that the plateau phase of the cycle exhibits an excess of features with large flux and that the global magnetic field configuration is strongly determined by the distribution of magnetic features in the photosphere. 
\item \emph{What mechanisms govern the evolution and decay of sunspot groups?} The decay of sunspot groups is investigated by simulating solar surface flows. By comparing simulations of flux dispersal with observations diffuse magnetic flux, the diffusion coefficient characterising the flows is constrained.
\end{itemize}
Our results exemplify the benefit of using automated systems for solar feature analysis. For the first time, a time-dependent comparison between the properties of magnetic features and flaring over the solar cycle is presented. Also novel is the comparison between feature properties and the global configuration of the Sun. Finally, the first direct comparison between the poleward propagation of dispersing sunspot flux and a data-driven simulation is presented. This work is a study of the emergence, evolution, and decay of the solar surface magnetic field.


%Sunspot groups usually emerge with a bipolar structure which may exhibit a complicated loop configuration in the corona above.  Flares are bursts of X-rays which emanate from the top and foot points of these loops. Flare productive sunspot groups tend to be large and complicated, but the precise mechanisms which cause flares are not known. They are often associated with eruptions of plasma into interplanetary space, known as coronal mass ejections and can affect space-borne humans and their technology.
%The global structure of the solar magnetic field is affected by the distribution of small magnetic structures visible in the photosphere. The global field has been studied using potential field source surface (PFSS) and global magnetohydrodynamic (MHD) models, but a direct connection with the distribution of localised magnetic structures has not been drawn. 
%Eventually the magnetic structures decay- primarily due to photospheric surface flows which act to disperse the magnetic fields. As these structures diffuse over the solar surface, coronal holes (regions of low density responsible for fast solar wind outflows) are observed to form in areas of weak unipolar magnetic field. These structures have important implications for the properties of the solar wind throughout the heliosphere.\\ 
%\emph{Aims:} This thesis is concerned with answering the following questions:\\
%$\bullet$ What are the magnetic conditions in sunspot groups that result in solar flares?\\
%$\bullet$ What are the mechanisms that govern the evolution and decay of sunspot groups?\\
%$\bullet$ What are the processes that govern flux transport on a global scale and how does this relate to the solar dynamo?\\
%$\bullet$ How does the distribution of sunspot groups on the solar surface affect the properties of the solarwind throughout interplanetary space?\\
%Our physical understanding of the solar interior is partially tied to the emergence, evolution, and decay of sunspots as well as small scale magnetic flux elements that result from sunspot decay. In addition to a attaining an understanding of the global solar dynamics, it may be possible to predict solar activity such as solar flares and coronal mass ejections with a better understanding of how these phenomena are related to the evolution and magnetic topology of sunspots. 
%emergence to decay. SMART applies a series of simple image processing techniques to photospheric magnetograms to detect and extract these features. By comparing consecutive sets of detections, features are tracked through time, allowing us to study the evolution of their magnetic characteristics. A global study of the dynamics of small scale magnetic flux elements will help in understanding large scale surface flows and dynamics of the global magnetic field of the Sun. The mechanisms of flaring may be better understood from studying the magnetic evolution of flare productive sunspot groups. The dispersion mechanisms of sunspot magnetic flux during their decay can be characterised by tracking the motion of diffuse flux over large time-scales. This thesis includes a description of sunspot emergence and evolution, the methods we use to try and answer the questions outined above, and the results of these investigations.

\end{abstracts}
%\end{abstractlongs}


% ---------------------------------------------------------------------- 
